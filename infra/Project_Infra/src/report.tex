

\documentclass[]{report}
\usepackage{algorithm}
\usepackage{algorithmic}


\title{}
\author{}


\begin{document}
	
	


\maketitle
\begin{center}
\textbf{3 differents ways to translate picture} 
\end{center}

\begin{itemize}
	
	\item moove each pixels one by one
	\item crop the zone which will still be in the picture after translation and paste it in a white picture
	\item use the openCV function "warpAffine" with the right parameters which allow us to translate with floating parameters.

\end{itemize}

\newpage

\begin{center}
	\textbf{Many different ways to estimate the translation parameters $p_x$ and $p_y$ assuming that they are discrete values} 
\end{center}

\begin{itemize}
	\item The greedy one : go through the whole picture and try each translation combinations and, for all of these, compute the error.\\
	
	\item A better one is locate the barycenter of both pictures, compare them and find two intervals (one for $p_x$and one for $p_y$). You will then be able to use the bruteforce algorithm described previously but only on these intervals.\\
\end{itemize}

/!\ insert picture here(the graphs)\\


The absolute error picture show us that the translation parameter is not an integer.\\

/!\ insert picture here( abs error int int)\\


\newpage

\begin{center}
	\textbf{How to find the more closest couple ($\^p_x,\^p_y$) to the translation parameters ? } 
	
\end{center}

\begin{algorithm}
	\caption{Calculate ($\^p_x,\^p_y$)}
	\begin{algorithmic} 
		\REQUIRE We already know ($p_x^{*},p_y^{*}$) 
		\FOR{$n \in [\![ 1, ... ]\!]$}
			\FOR{$x \in [\![ -10, 10 ]\!]$}
				\FOR{$y \in [\![ -10, 10 ]\!]$}\\
					\STATE
				
					\STATE E[i] $\leftarrow$ error at point P ($p_x^{*}+\frac{x}{10^{-n}},p_y^{*}+\frac{y}{10^{-n}}$)
					\STATE P[i] $\leftarrow$ ($p_x^{*}+\frac{x}{10^{-n}},p_y^{*}+\frac{y}{10^{-n}}$)
					\STATE
					
					
					
		\ENDFOR
		\ENDFOR
		\STATE ($p_x^{*},p_y^{*}$) $\leftarrow$  P[ index(max(E)) ]
		\STATE E $\leftarrow$ []
		\ENDFOR
		
	\end{algorithmic}
\end{algorithm}



After this we obtain as absolute error picture :

/!\ insert picture here( abs error float)\\


\end{document}          

